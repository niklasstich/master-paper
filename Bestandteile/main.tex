\chapter{Was ist Monero?}
Monero ist eine dezentralisierte Proof-of-Work (PoW) Kryptowährung, basierend auf einer gleichnamigen, verteilten Blockchain\cite[vgl. Abstract]{Koe2020}.
Die wichtgsten Ziele der Währung Monero sind die Ermöglichung anonymer, privater Transaktionen, Sicherheit der Benutzer sowie Nichtnachvollziehbarkeit von Addressen und Transaktionen im Ledger.
Monero gehört somit zu den sogenannten ``Privacy Coins''.

\section{Privacy Coins}
Als Privacy Coins werden solche Kryptowährungen bezeichnet, welche den Benutzern vollständige Anonymität über ihre Transaktionen, den beteiligten Personen sowie transferierten Beträgen
und in Folge Verschleiern von Besitztümern vor Dritten garantieren.
Sie stehen hiermit im Kontrast zu anderen, konventionelleren Kryptowährungen wie Bitcoin oder Ethereum, welche ihren Benutzern lediglich Pseudonymität versprechen\cite{Vermaak2021} (mehr dazu in Kapitel \ref{reasons}).
Andere bekannte Privacy Coins neben Monero sind beispielsweise Dash und Zcash, beide sind Bitcoin-Forks, Beam oder Grin. Diese verwenden unterschiedliche Technologien um ihren Benutzern Anonymität anzubieten.\\
Aufgrund diverser externer Faktoren, unter anderem gesetzliche Faktoren auf die in Kapitel \ref{legal} eingegangen wird, besitzen die meisten Privacy Coins nur marginale Anteile der Gesamtmarktkapitalisierung im Bereich der Kryptowährungen, 
mit Monero als Anführer auf ungefähr Platz 25 mit ca. 4 Mrd. US-Dollar Marktkapitalisierung (Stand Juli 2021)\cite{CoinMarketCap2021}.

\section{Untracability}


\section{Unlinkability}

\chapter{Kryptowährungen und Anonymität - ein Widerspruch?}

\chapter{Legitime Gründe für anonyme Zahlungen oder: Warum Monero wichtig ist} \label {reasons}

\chapter{Historie des Monero-Projekts}

\chapter{Beispieltransaktionen in Monero}

\chapter{Wie funktioniert Monero?}

\section{Elliptische Kurven und Schlüssel}

\section{One-time Addresses}

\section{Ring Signatures}

\section{RingCT}

\section{RandomX}

\chapter{Gesetzliche Einschränkungen} \label{legal}

\section{Währungsmonopol und Dezentralisierung}

\section{Steuerrechtliche Relevanz}

\section{Einfluss auf Broker und Nutzer}