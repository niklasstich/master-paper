\chapter{Was ist Monero?}
Monero ist eine dezentralisierte Proof-of-Work (PoW) Kryptowährung, basierend auf einer gleichnamigen, verteilten Blockchain\cite[vgl. Abstract]{Koe2020}.
Die wichtgsten Ziele der Währung Monero sind die Ermöglichung anonymer, privater Transaktionen, Sicherheit der Benutzer sowie Nichtnachvollziehbarkeit von Addressen und Transaktionen im Ledger.
Monero gehört somit zu den sogenannten ``Privacy Coins''.

Als Privacy Coins werden solche Kryptowährungen bezeichnet, welche den Benutzern vollständige Anonymität über ihre Transaktionen, den beteiligten Personen sowie transferierten Beträgen
und in Folge Verschleiern von Besitztümern vor Dritten garantieren.
Sie stehen hiermit im Kontrast zu anderen, konventionelleren Kryptowährungen wie Bitcoin oder Ethereum, welche ihren Benutzern lediglich Pseudonymität versprechen\cite{Vermaak2021} (mehr dazu in Kapitel \ref{anonymity}).
Andere bekannte Privacy Coins neben Monero sind beispielsweise Dash und Zcash, beide sind Bitcoin-Forks, Beam oder Grin. Diese verwenden unterschiedliche Technologien um ihren Benutzern Anonymität anzubieten.\\
Aufgrund diverser externer Faktoren, unter anderem gesetzliche Faktoren auf die in Kapitel \ref{legal} eingegangen wird, besitzen die meisten Privacy Coins nur marginale Anteile der Gesamtmarktkapitalisierung im Bereich der Kryptowährungen, 
mit Monero als Anführer auf ungefähr Platz 25 mit ca. 4 Mrd. US-Dollar Marktkapitalisierung (Stand Juli 2021)\cite{CoinMarketCap2021}.


\chapter{Kryptowährungen und Anonymität - ein Widerspruch?} \label{anonymity}
Zunächst einmal muss der Begriff Anonymität genauer definiert werden. Wir können eine Kryptowährung dann als anonym bezeichnen,
wenn die folgenden Punkte gegeben sind: 
\begin{enumerate}
    \item Dritte können Benutzer des Systems nicht von anderen Nutzern unterscheiden, die ähnliche Aktionen durchführen (z.B. Transaktionen senden und empfangen)\cite{Bleumer2011b}.
    \item Transferierte Beträge in Transaktionen sind für Dritte nicht einsehbar.
    \item Transaktionen erfüllen das Kriterium der ``Unlinkability'': 
\end{enumerate} 

Für Leser, welche nur mit der Funktionsweise ``traditioneller'' Kryptowährungen wie Bitcoin oder Ethereum vertraut sind und Monero lediglich oberflächlich kennen, 
mag es zunächst kontraintuitiv oder eventuell sogar widersprüchlich klingen eine anonyme Blockchain zu kreieren, wo sich klassische Blockchains doch gerade dadurch auszeichnen,
dass jeder Teilnehmer des Netzwerks jede bisher getätigte Transaktion verifizieren kann. Hierfür ist es typischerweise nötig, sowohl den Sender als auch den Empfänger sowie
den Betrag jeder Transaktion zu kennen, um die Kette von Transaktionen seit dem Genesisblock nachzuvollziehen. Diese Anforderung ergibt sich aus dem Fakt, dass alle erhaltenen Transaktionsoutputs nur einmal
ausgegeben werden dürfen (Verhindern von Double-Spending). Aus diesem Grund wird der Ledger, welcher alle Transaktionsdaten enthält,
dezentral durch alle Teilnehmer des Netzwerks gespeichert.\footnote{Auf den Unterschied zwischen Full Nodes und Lightweight Clients, welche nicht selbst eine Kopie der vollständigen
Blockchain halten, soll hier nicht weiter eingegangen werden, hierzu sei z.B. auf Bitcoin SPV und die diversen web- und mobilebasierten Bitcoinwallets verwiesen (bspw. https://bitcoin.org/en/wallets/mobile/android/bitcoinwallet/). }\\

Monero stellt dieses Konzept nun auf den Kopf, indem Dritte weder Sender, Empfänger noch Betrag von Transaktionen kennen. Man mag nun naiverweise annehmen, dass Transaktionen
in einer solchen Kryptowährung nicht verifizierbar sind. Durch die clevere Anwendung verschiedener kryptographischer Verfahren ist es jedoch trotz Anonymität dennoch möglich, Transaktionen
auf Monero zu verifizieren und somit die Integrität der Währung zu gewährleisten. Auf die verwendeten Technologien wird in Kapitel \ref{tech} eingegangen.

\chapter{Legitime Gründe für anonyme Zahlungen oder: Warum Monero wichtig ist} \label {reasons}

\chapter{Historie des Monero-Projekts}

\chapter{Beispieltransaktionen in Monero}

\chapter{Wie funktioniert Monero?} \label{tech}

\section{Elliptische Kurven und Schlüssel}

\section{One-time Stealth Addresses}

\section{Ring Signatures}

\section{RingCT}

\section{RandomX}

\chapter{Gesetzliche Einschränkungen} \label{legal}

\section{Währungsmonopol und Dezentralisierung}

\section{Steuerrechtliche Relevanz}

\section{Einfluss auf Broker und Nutzer}