\newpage

\vspace*{1cm}

\begin{center}
    \textbf{Kurzzusammenfassung}
\end{center}

\vspace*{1cm}

\noindent 
Diese Seminararbeit bietet einen Überblick über die anonyme Kryptowährung Monero, der zugrundeliegenden Technologien der eingesetzten Blockchain und externe sowie gesetzliche Faktoren bezüglich der gebotenen Anonymität. 
Zunächst wird die Währung konzeptionell eingeführt, dann wird kurz untersucht ob Kryptowährungen und Anonymität einen Widerspruch in sich selbst darstellen. Dann werden einige legitime Gründe für den Einsatz anonymer Währungen geboten.
Nach einem kurzen Überblick über die Historie von Monero und der Beleuchtung einiger beispielhafter Transaktionen auf der Blockchain
werden die kryptographischen Primitive Elliptische Kurven, Ring Signatures, RingCT und der Mining-Algorithmus RandomX technisch analysiert und aufgearbeitet.
Abschließend werden einige mögliche gesetzliche Einschränkungen bezüglich des Währungsmonopols von Staaten, Auswirkungen auf Broker und steuerrechliche Relevanz betrachtet.