\documentclass[12pt,oneside,listof=totoc,paper=a4,headings=small]{scrbook}
% Nützliche Packages für die Gestaltung und allgemeine Konfiguration des Dokuments
% -----------------------------------

% Allgemeine Formatierungen
\usepackage{ngerman}				% neue deutsche Rechtschreibung
\usepackage[utf8]{inputenc} 		% Umlaute im Text
\usepackage[T1]{fontenc}
\usepackage{xspace}                 % Vermeidung von "ineinanderfallenden f's", wie z.B. bei Schifffahrt
\usepackage{url}		            % korrekte Anzeige/Umbruch von URLs
\usepackage{listings}               % z.B. nützlich zum Einbinden von Quellcode
\usepackage{hyperref} 				% für Hyperlinks in PDF-Dokumenten 
\usepackage{lmodern}
\usepackage{enumerate}

% Kopfzeile
\usepackage[headsepline,manualmark]{scrlayer-scrpage}
\clearpairofpagestyles
\ohead{\pagemark}
\ihead{\headmark}
\automark{chapter}
\pagestyle{scrheadings}

% Seitenspiegel
\usepackage[left=25mm,right=20mm,top=25mm,bottom=25mm]{geometry}

% Literatur
\usepackage[square,sort,comma,numbers]{natbib}         % Deutsches Literaturpackage

% Grafiken
\usepackage{graphicx} 				% Grafiken einfügen (pdf,png - aber jpg vermeiden)
\graphicspath{{./Bilder/}}          % Pfad zu den Bildern

% Tabellen
\usepackage{booktabs} 				% bessere Gestaltung von Tabellen
\usepackage{longtable} 				% für bessere Tabellen über mehrere Seiten

% Koma-Script Kompatibilität
\usepackage{scrhack}

% Das Dokument selbst mit seinen Bestandteilen
% --------------------------------------------
\begin{document}
\frontmatter 
% ----------------------------------------------
    % Titelseite soll keine Kopf oder Fußzeile haben
\thispagestyle{empty}



\vspace*{-20mm}
\begin{flushright}
\includegraphics[width=0.5\textwidth]{Bilder/LogoHS.png}
\end{flushright}


\vspace*{2cm}

% Alle Elemente sollen zentriert sein
\begin{center}
% Art der Arbeit => (Bachelorarbeit, Masterarbeit, Seminararbeit)
{\Large \textbf{Seminararbeit}}\\ 

\vspace*{1cm}

{\large Studiengang Informatik (Master)\\[1mm]}

\vspace{1cm}

% Titel der Arbeit 
{\Large \bfseries Seminararbeit im Seminar \\ 
	Blockchain, Kryptowährungen und Smart Contracts\\
	zum Thema Monero\\}


\vspace{1.5cm}

% Name des/der Autors/Autoren
{\large Niklas Stich}\\[40mm]

\end{center}

\vfill

% Aufgabensteller, Kontaktdaten und Abgabetermin
\parbox{120mm}{
\begin{tabbing}
Aufgabensteller/Prüfer \hspace{.7cm} \= Prof. Dipl.-Inf. N. Steger\\
Arbeit vorgelegt am                  \> 09.07.2021\\
durchgeführt in der                  \> Fakultät Informatik\\[4mm]
% Die Nennung des Betreuers ist freiwillig und mit diesem abzustimmen
\end{tabbing}
}

 				    % Titelblatt
    \newpage

\vspace*{1cm}

\begin{center}
    \textbf{Kurzzusammenfassung}
\end{center}

\vspace*{1cm}

\noindent 
Diese Seminararbeit bietet einen Überblick über die anonyme Kryptowährung Monero, der zugrundeliegenden Technologien der eingesetzten Blockchain und externe sowie gesetzliche Faktoren bezüglich der gebotenen Anonymität. 
Zunächst wird die Währung konzeptionell eingeführt, dann wird kurz untersucht ob Kryptowährungen und Anonymität einen Widerspruch in sich selbst darstellen. Dann werden einige legitime Gründe für den Einsatz anonymer Währungen geboten.
Nach einem kurzen Überblick über die Historie von Monero und der Beleuchtung einiger beispielhafter Transaktionen auf der Blockchain
werden die kryptographischen Primitive Elliptische Kurven, Ring Signatures, RingCT und der Mining-Algorithmus RandomX technisch analysiert und aufgearbeitet.
Abschließend werden einige mögliche gesetzliche Einschränkungen bezüglich des Währungsmonopols von Staaten, Auswirkungen auf Broker und steuerrechliche Relevanz betrachtet.  	% Abstract
    \tableofcontents 					            % Inhaltsverzeichnis
    \clearpage
\mainmatter 						% die einzelnen Kapitel, bei Bedarf weitere *.tex Dateien erzeugen und hier einbinden
    \chapter{Was ist Monero?}
Monero ist eine dezentralisierte Kryptowährung, basierend auf einer gleichnamigen, verteilten Blockchain.\cite{Koe2020}

\section{Privacy Coins}

\section{Untracability}

\section{Unlinkability}

\chapter{Kryptowährungen und Anonymität - ein Widerspruch?}

\chapter{Legitime Gründe für anonyme Zahlungen oder: Warum Monero wichtig ist}

\chapter{Historie des Monero-Projekts}

\chapter{Beispieltransaktionen in Monero}

\chapter{Wie funktioniert Monero?}

\section{Elliptische Kurven und Schlüssel}

\section{One-time Addresses}

\section{Ring Signatures}

\section{RingCT}

\section{RandomX}

\chapter{Gesetzliche Einschränkungen}

\section{Währungsmonopol und Dezentralisierung}

\section{Steuerrechtliche Relevanz}

\section{Einfluss auf Broker}
% ----------------------------------------------
    \listoffigures  					 	        % Abbildungsverzeichnis
    \clearpage
    \listoftables						            % Tabellenverzeichnis 
    \clearpage
\backmatter 

% Bitte Zitierstil mit Betreuer absprechen! Möglich sind z.B.
%\bibliographystyle{natdin}
%\bibliographystyle{abbrvdin}			
\bibliographystyle{alphadin}		% Zitierstil des Literaturverzeichnisses

\bibliography{./Literatur/quellen}	% Literatur (BibTex) einbinden 
\input{./Bestandteile/erklaerung} 	% Erklärungen - Unterschreiben nicht vergessen!

\end{document}